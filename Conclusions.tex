\section{Conclusions}

A range of methods were employed in CTF3 to setup and optimise the beams. 
Some of them are very specific to drive beam recombination.
Extensive studies were carried out and the results were pres

The ability of measuring and controlling dispersion in the different beam lines has been demonstrated.
A series of examples has proven the potential of using dispersion not only for beam steering (DFS and DTS),
but also as a mean for optics optimisation. 
In the linac dispersion was below 5~mm and 1~mm in respectively horizontal and vertical planes.



Beam emittance is one of the principal figures of merit.
Table~\ref{tab:emittancesummary} lists the achieved emittances along the machine.

\begin{table}[h]
 \centering
  \begin{tabular}{rcccc}
    \hline
    
                                 & \multicolumn{2}{c}{Factor 8 (1.5~GHz)} & \multicolumn{2}{c}{Factor 4 (3~GHz)} \\
    {\multirow{3}{*}{Location} } & \multicolumn{2}{c}{[mm mrad]}          & \multicolumn{2}{c}{[mm mrad]}        \\
                                 & H     &    V                           &      H         & V                \\
    \hline \hline
                   DL injection  & 57   & 76                              &  66 &  75 \\
                   DL extraction & 120  & ??                              & -   &  - \\
                   CR extraction & 243  & 120                             & 148 &  91 \\
                   CLEX          & 420  & 122                             & 173 &  96 \\

    \hline
  \end{tabular}
\caption{Emittance measurements for factor 4 and factor 8 combined drive beam.}
\end{table}
%% DL H 120 http://elogbook.cern.ch/eLogbook/event_viewer.jsp?eventId=2225951


The obtained emittance evolution of the uncombined beam approximately agree with the simulations. 
Injector produced between 40 and 50~mm~mrad. It was well preserved until the end of the linac.
%Even with the Stretching chicane the emittance at the injection of the Delay Loop and 
%the Combiner Ring was at the order of 

For the vertical plane the goal of 150~mm~mrad for the combined beam was achieved and 
for factor 4 in the horizontal plane as well. For factor 8, 240~mm~mrad was achieved behind the Combiner Ring.

\todo[inline]{I can not find for the time being factor 4 with 1.5GHz below 250}
% the last I found is 250 http://elogbook.cern.ch/eLogbook/event_viewer.jsp?eventId=2228297

The evolution of the emittance along the machine for the recombined beam 
was worse than expected in the design.
The reasons were understood thanks to more detailed calculations done only during machine operation.
First, the time variable injection bump of the Combiner Ring was not fully achromatic 
creating dispersion wave for the first injected sub-pulse.
Second, the optics of the Delay Loop and of the TL2 line had too small momentum acceptance.
%were heavily constrained by the pre-existing building layout.



For the CTF3 Design Report adequate solutions were found. However, due to cost reasons,
and also due to evolving design of the machine, they were modified.
In particular, the assumed Delay Loop lattice composed of 34 compact quadrupoles, 
which was replaced by 20 quadrupoles. 
The resulting optics yielded large non-linear dispersion leading to emittance growth 
of the beam with 0.6\%~r.m.s. energy spread. 
Additionally, the implemented quadrupoles that were recycled from decommissioned machines
turned out to have insufficient precision. 
%Because they were much larger, 
The remaining space allowed to install only a sub-optimal number of BPMs and of orbit correctors. 
In consequence, the orbit control was very difficult and implementation of the sextupolar corrections,
which was hoped to limit the emittance growth, turned out to be impossible.
Finally, the optics corrections were difficult because all the magnets were powered in pairs.

Another item hampering the optimisations was the machine stability. 
In the time scales required for a set of measurements and corrections its characteristic was fluctuating 
such that the applied corrections were not accurate.
%increasing the error bars above the required minimum.
During the initial years it was virtually impossible to preform any emittance optimizations 
and only after implementation multiple feedback systems in 2012-2013 the situation become acceptable.
Still, again for the Delay Loop, the jittering septa power supply could not be fixed.
The Delay Loop optics turned out to be particularly sensitive to this jitter and 
there was no flexibility to reduce it. 
Finally, the total beam time when all 3 SHB 1.5~GHz sources were available was relatively short
and the operational experience with this beam was much smaller than with the 3~GHz beam. 


