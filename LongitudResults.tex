\section{Bunch length and energy spread results}

The drive beam injector was delivering the beam with the energy spread of 0.6\% 
and it was preserved by the linac. 
However, the phase switches affected the momentum distribution and it was 0.X\%.
The corresponding bunch length was measured was XX~ps.


%%%%%%%%%%%%%%%%%%%%%%%%%%%%%%%%%%%%%%%%%%%%%%%%%%%
% Davide paper MOPAB114
%\section{Drive Beam}

%%%%%%%%%%
\subsection{Indirect measurement}
{\color{red} All values here to be confirmed.}
In the spectrometer at girder 4 we normally measured $E_0\approx18$~MeV and 
energy spread (FWHM) $\delta E_0 \approx 5 \% = 0.9$~MeV.
In the spectrometer at girder 10 we normally measured $E_0\approx64$~MeV and 
energy spread (FWHM) $\delta E_0 \approx 2.8 \% = 1.79$~MeV.
In CTS we measured about $E \approx 135$~MeV and energy spread 
(FWHM) $\delta E \approx 1.5 \% = 2.03$~MeV.
%
In absolute number we see an increase of energy spread from the injector to the end of the linac.
Let us assume that the acceleration was always on crest in the whole linac, i.e. $\phi = 0$ and 
so we can use directly Eq.~\ref{eq:energySpreadIncreaseOnCrest} to extract the bunch length:
%
\begin{align}
\cos\left(\frac{\delta \phi}{2}\right) &= 1 - \frac{\Delta \delta E}{E_{max}}.
\label{eq:simplestBunchLengthMeas}
\end{align}
%
We can apply Eq.~\ref{eq:simplestBunchLengthMeas} for the three different cases: CTS--04; CTS--10; 10--04.
In these three cases the bunch length $\delta \phi$ results to be: 15.7, 9.4; 22.6~deg FWHM.
%
The RF at CTF3 has a frequency of 3 GHz. This means that 1 deg is equivalent to $0.926$~ps, 
so the previous results are approximately equivalent in~ps.
Note that the nominal bunch length is expected to be of the order of 1~mm~r.m.s, 
which is about $2.355 \times 3.333 = 7.8$~ps~FWHM.



%%%%%%%%%%
\subsection{Direct measurement}
%
Figure~\ref{fig:scanDriveBeam} shows a scan performed at the end of the Drive Beam linac in 
the CTS spectrometer.
%
\begin{figure}[htb]
   \centering
   \includegraphics*[width=0.66\columnwidth]{MOPAB114f2_v2.eps}
   \caption{Energy (blue) and Energy spread (red) as a function the last two 
            DB linac accelerating structures. Dashed are the expected values obtained from 
            the fitted bunch length.}
   \label{fig:scanDriveBeam}
\end{figure}
%
From the fit discussed in the previous section~\ref{sec:rfphasescans} it was possible to measure the following parameters:
%
\begin{itemize}
\item
$\delta\phi_{s} = 3.3 \pm 0.1$ ps r.m.s.; 
 \item
$\delta E_0 = 0.68 \pm 0.01$ \%;
\item 
$\phi_{beam} = -1.2 \pm 0.1$ deg@3GHz;
\item 
$C = 0.15 \pm 0.01$ \% r.m.s. / ps 
\end{itemize}
%
Note that 3.3~ps matches the nominal bunch length value.

%%%%%%%%%%%%%%%%%%%%%%%%%%%%%%%%%%%%%%%%%%%%%%%%%%%
\subsection{Probe Beam}
% measured with a x5; 15 A DB  on Wed 14th Dec 2016; file_name = 'RFdata_2015.mat'
Figure~\ref{fig:scanProbeBeam} shows the result of a similar scan on the Probe beam.
In this case the arrival phase of the Probe Beam has been varied with respect to the DB-generated RF. 
Thanks to the higher frequency of RF it was possible to scan over the full 360 degrees at 12 GHz, 
giving a more precise fit.
%
\begin{figure}[htb]
   \centering
   \includegraphics*[width=0.66\columnwidth]{MOPAB114f3.eps}
   \caption{Energy (blue) and Energy spread (red) as a function the Probe Beam phase with respect to 
            the DB-generated RF. Dashed are the expected values obtained from the fitted bunch length.}
   \label{fig:scanProbeBeam}
\end{figure}
%
The fitted values in this case (for the Probe Beam) are:
%
\begin{itemize}
\item
$E_0 = 198.7 \pm 0.1$ MeV; 
\item
$\Delta E_{max} = 26.3 \pm 0.1$ MeV; 
%
\item
$\delta\phi_{s} = 2.69 \pm 0.02$ ps r.m.s.; 
 \item
$\delta E_0 = 0.46 \pm 0.04$ \%;
\item 
$\phi_{beam} = 5.6 \pm 0.4$ deg@12GHz;
\item 
$C = -0.08 \pm 0.08$ \% r.m.s. / ps 
\end{itemize}
%
%
Also in this case the value obtained is compatible with the expected Probe Beam bunch length.


