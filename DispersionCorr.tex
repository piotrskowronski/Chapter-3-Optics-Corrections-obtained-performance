\subsection{Dispersion corrections}

In CTF3 two methods were tried for correcting dispersion errors:
with quadrupoles, using dedicated knobs that modifed only dispersion without changing beta functions,
and Dispersion Free Steering.  
At several occations the two were compared and in CTF3 DFS was always much more efficient.
Global quadrupolar corrections that would correct simoultaneously dispersion and beta functions
were difficult to find because the beta functions were very sensitive to focusing changes
and often quadrupoles were powered in series. Also, until Phase Space Painting was 
made fully operational, precise beta function measurements were possible only 
at few locations (with quadrupolar scans using profile monitors).
Therefore, correcting first beta functions and dispersion idependently with DFS
was naturally easier. Of course, there was a risk that created orbit was
misaligned what would make sextupolar corrections tricky.

\tb{Taken from Davide's LINAC16 paper}

Figure~\ref{fig:linacDFS} shows the result of applying DFS in the CTF3 linac. 
%
\begin{figure}[!htb]
   \centering
   \includegraphics*[width=0.6\columnwidth]{MOPRC008f2.eps}
   \caption{Horizontal dispersion along the linac at CTF3.
   Black is the design dispersion.
   Red and blue are the actual dispersions measured by changing the beam energy before and after DFS respectively.}
   \label{fig:linacDFS}
\end{figure}
%
The correction was performed by first \emph{measuring} directly on the machine the response matrix of 
all dipole correctors in the linac.
Note the reduction in spurious dispersion below 5~mm.
At the same time a similar correction was performed in the vertical plane reducing 
the vertical dispersion from about 10 mm to less than 1 mm \cite{bib:DavideThesis}.
The effect of those corrections was extremely beneficial for the final beam quality.
Table~\ref{tab:TwissResultLinac} shows the Twiss parameters of the beam measured at 
the end of the linac before and after DFS in the two planes.
It is remarkable that the observed emittance was reduced by more than 15~\% in both planes, 
which is proof of the effectiveness of DFS%
\footnote{The asymmetry between the horizontal and vertical emittances was probably due to errors at the source, lately corrected by other means.}. 
%
\begin{table*}[htbp]
\centering
\caption{
Transverse Twiss parameters of the beam measured at the end of the Drive Beam linac at CTF3 before and after DFS in the linac.
Also shown are the nominal Twiss parameters for the ideal machine.}
\begin{tabular}{r c c c c c c}
\hline
				& $\beta_x$  [m]	&  $\alpha_x$ 		&  $\epsilon_{Nx}$	 [$\mu$m]	 & $\beta_y$  [m]	&  $\alpha_y$ 		&  $\epsilon_{Ny}$	 [$\mu$m]	  \\
\hline
Nominal Twiss 	& $8.4$ 			& $-0.8$ 			& -- 						& $13.5$ 			& $-0.4$ 			& -- \\
Before DFS		& $9.2 \pm 0.4$ 	& $-0.7 \pm 0.1$ 	& $\mathbf{63 \pm 1}$ 				& $11.3 \pm 1.2$	& $-0.1 \pm 0.1$	&  $\mathbf{129 \pm 8}$ \\
After DFS			& $8.7 \pm 0.4$ 	& $-0.5 \pm 0.1$ 	& $\mathbf{52 \pm 1}$ 				& $10.3 \pm 1.0$	& $-0.1 \pm 0.1$	&  $\mathbf{102 \pm 5}$ \\
\hline
\end{tabular}
\label{tab:TwissResultLinac}
\end{table*}
%


DFS was performed also in the DBRC, but clearly only in the vertical plane where 
no dispersion is expected by design.
In the horizontal plane, where dispersion is non-zero by design in most location, 
DTS has been tested.
Note that for DTS one needs first to know the target dispersion.
Naively one could try to target the design dispersion, 
however any BPM calibration issue or a wrong set up of the quadrupoles strength would 
drive the correction to an undesired state.
Here the concept of ``nominal'' dispersion previously introduced becomes extremely useful.

Figure~\ref{fig:CRDTS} shows the result of a DTS attempt in the CR at CTF3.
%
\begin{figure}[!htb]
   \centering
   \includegraphics*[width=0.6\columnwidth]{MOPRC008f3.eps}
   \caption{Horizontal dispersion along the CR at CTF3.
   Black is the design dispersion.
   Green is the ``nominal'' dispersion measured by scaling the bending magnets of the ring, and it was used as target for DTS.
   Red and blue are the actual dispersions measured by changing the beam energy before and after DTS respectively.}
   \label{fig:CRDTS}
\end{figure}
%
Note the discrepancy at $s \approx 67$ m between the design dispersion (black) and 
the measured ``nominal'' dispersion (green) due to a mis-calibration of a BPM. 
It is clear that if one would target the design dispersion at this location 
one would have driven the line toward an undesired set-up.
For this correction only dipole correctors inside the CR were used, 
therefore in the first part of the ring DTS does not have enough degrees of freedom,
but the correction starts to be effective in the second half of the ring.

DTS is a promising technique, but further experimental verification are needed to 
prove its effectiveness in improving the Drive Beam recombination quality.

%%%%%%%%%%%%%%%%%%%%%%%%%%%%%%%%%%%%%%%%%%%%%
\subsubsection{Dispersion for machine set up and optimisation}
%
One of the recent improvements of the recombination process at CTF3 was 
the optimisation of the DL optics in order to reduce the outgoing non-linear dispersion \cite{gambaIPAC16}.
The ability of measuring non-linear dispersion turned out to be useful as a verification of the improvement.
Figure~\ref{fig:DLnewOptics} shows a scatter plot of consecutive beam shots with different energies at 
the first BPM after DL for the two different optics.
%
\begin{figure}[!htb]
   \centering
   \includegraphics*[width=0.6\columnwidth]{MOPRC008f4.eps}
   \caption{Comparison of non-linear dispersion at the first BPM after the DL for two different DL optics.
   Scatter plot of the mean position recorded at the BPM versus beam energy variation.}
   \label{fig:DLnewOptics}
\end{figure}
The second order dispersion is clearly visible for both DL optics, but the effect is sensibly reduced for the new one (blue).

Another important use of the dispersion as indicator of the quality of set-up is the use of 
the ``nominal'' dispersion measurement previously introduced.
Since such a measurement is not affected by misalignments, it gives a direct measurement of 
the correctness of the quadrupole and relative dipole strengths.
Figure~\ref{fig:CRfirstarc} shows one of these measurements in the first arc of the CR.
%
\begin{figure}[!htb]
   \centering
   \includegraphics*[width=0.6\columnwidth]{MOPRC008f5.eps}
   \caption{Orbit response at the BPMs in the first arc of CR while scaling the ring dipoles.
   Black is the design dispersion.
   Coloured are actual measurements: the initial status (red) and scaling the arc quadrupoles by 
   -1\% (yellow), +2\% (purple), +3\%(green).}
   \label{fig:CRfirstarc}
\end{figure}
%
Note that in the middle of the arc one expects a dispersion close to zero, while the initial measurement (red) 
was measuring a ``nominal'' dispersion sensibly different from zero.
By scaling up the quadrupoles in the arc the pattern got closer to the design (e.g. green).
The improvement could be seen also observing the variation of the ring $R_{56}$.
As one expects to measure the nominal $D_x$ while scaling the bending magnets,
then one should be able to reveal at the same time the ``nominal'' $R_{56}$.
The optics of the CR is meant to be isochronous \cite{bib:CTF3DesignReport}.
For the purpose of beam recombination two RF deflectors are installed around the CR injection. 
After one turn in the ring the beam is expected to cross the deflectors on zero-crossing. 
Clearly if $R_{56}$ is non-zero, any variation of path length while scaling the bending magnets results in 
a visible bump in the orbit, which is seen by the dispersion monitor application as actual ``nominal'' dispersion.
Figure~\ref{fig:CRRFbump} shows this effect during the optimisation of the arc quadrupole strengths presented in 
Figure~\ref{fig:CRfirstarc}.
%
\begin{figure}[!htb]
\centering
   \begin{tikzpicture}
      \node[anchor=south west,inner sep=0] (image) at (0,0) {\includegraphics*[width=0.6\columnwidth]{MOPRC008f6.eps}};
      \begin{scope}[x={(image.south east)},y={(image.north west)}]
        \draw [dashed, thick, blue] (0.24,0.17) -- (0.24,0.9);
        \draw [dashed, thick, blue] (0.80,0.17) -- (0.80,0.9);
        %
        \node[below, align=center, blue, fill=white] at (0.24,0.9) {\footnotesize RF def. 2};
        \node[below, align=center, blue, fill=white] at (0.80,0.9) {\footnotesize RF def. 1};
      \end{scope}
   \end{tikzpicture}
   \caption{Orbit response at the BPMs around the RF deflectors of the CR while scaling the ring dipoles for 
            the same set-ups of Figure~\ref{fig:CRfirstarc}.
            The additional blue measurement was performed with the initial quadrupole strengths but 
            without RF into the deflectors.}
   \label{fig:CRRFbump}
\end{figure}
%
Note that also in terms of $R_{56}$ by scaling up the arc quadrupoles the optics got closer to nominal.
As a proof that the effect was really given by the lengthening of the beam path, 
note that when RF was removed from the deflector the effect disappeared (see blue curve in Figure~\ref{fig:CRRFbump}).

From Figure~\ref{fig:CRRFbump} one can be more quantitative: the orbit excursion expected with the RF bump can be written as:
\begin{equation}
\Delta x \approx R_{56} \frac{\Delta p}{p_0} \frac{2 \pi}{\lambda_{RF}} x_{max}.
\label{eq:orbitErrorDueToR56}
\end{equation}
%
where $\lambda_{RF}$ is the RF wavelength and $x_{max}$ is the maximum orbit excursion expected when 
the beam is crossing the cavities on crest. 
By scaling the bending magnets one actually measures the overall linear coefficient of 
Eq.~\ref{eq:orbitErrorDueToR56} with respect to 
$\Delta p/p_0$.
By knowing that $x_{max}\approx 25$ mm; $\lambda_{RF}\approx 10$ cm one can than estimate 
$R_{56} \approx 0.16$~m before the correction (red) and $R_{56} \approx 0.04$~m after the correction (green).



%%%%%%%%%%%%%%%%%%%%%%%%%%%%%%%%%%%%%%%%%%%%%
\subsubsection{Conclusions}
%
\todo[inline]{Move to final conclusions}
The ability of measuring and controlling dispersion in the different beam lines has been demonstrated.
A series of examples has proven the potential of using dispersion not only for beam steering (DFS and DTS),
but also as a mean for optics optimisation.

